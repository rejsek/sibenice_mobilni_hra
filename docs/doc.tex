\documentclass[a4paper,12pt]{article}
\usepackage[czech]{babel}
\usepackage[utf8]{inputenc}
\usepackage[T1]{fontenc}
\usepackage{geometry}
\geometry{margin=2.5cm}
\usepackage{graphicx}
\usepackage{hyperref}
\usepackage{enumitem}
\usepackage{titlesec}
\titleformat{\section}{\normalfont\Large\bfseries}{\thesection.}{1em}{}
\titleformat{\subsection}{\normalfont\large\bfseries}{\thesubsection.}{1em}{}
\setlength{\parskip}{0.5em}

\title{\textbf{Dokumentace mobilní hry Šibenice}}
\author{}
\date{}

\begin{document}
\maketitle

\section{Úvod}
Mobilní hra \textbf{Šibenice} je interaktivní vzdělávací hra vytvořená pomocí frameworku React Native s využitím Expo a Tailwind CSS. Hra je založena na klasické slovní hádance Šibenice, kde uživatel hádá tajné slovo dle tematického okruhu a obtížnosti.

Cílem hry je uhodnout hledané slovo s co nejmenším počtem chyb a získat tak maximální počet hvězd. Hra je lokalizovaná pro české uživatele a podporuje tmavý i světlý režim.

\section{Použité technologie a struktura projektu}

\begin{itemize}[leftmargin=*]
  \item \textbf{React Native (Expo)} – cross-platformový framework pro vývoj mobilní aplikace pro iOS a Android.
  \item \textbf{TypeScript} – silně typovaná verze JavaScriptu pro bezpečnější kód.
  \item \textbf{Tailwind CSS (twrnc)} – utility-first framework pro styling přímo v komponentách.
  \item \textbf{AsyncStorage} – pro lokální ukládání nastavení, skóre a preferencí.
  \item \textbf{react-native-chart-kit} – pro vykreslování koláčových grafů v sekci herního postupu.
\end{itemize}

Struktura aplikace je rozdělena do složek dle obrazovek (např. /settings.tsx, /game.tsx), sdílených dat (např. words.json, buttons.json) a komponent (např. TreeView, imageMap). 

\section{Herní mechanika a logika}

\subsection{Princip hry}
Uživatel si vybere téma a obtížnost a poté si zvolí jednu z dostupných úrovní. Každá úroveň odpovídá jednomu slovu, které musí uživatel uhodnout zadáváním písmen. Po částečném nebo úplném uhodnutí slova se vypisuje aktuální stav slova a chyb.

\subsection{Skóre a hvězdy}
Každé slovo má přidělený počet hvězd podle úspěšnosti (počet chyb):
\begin{itemize}
  \item 3 hvězdy – bez chyby
  \item 2 hvězdy – max. 2 chyby
  \item 1 hvězda – max. 4 chyby
  \item 0 hvězd – příliš chyb nebo vypršel čas
\end{itemize}

\subsection{Zvláštní pravidla}
\begin{itemize}
  \item Volitelné omezení počtu pokusů nebo časový limit (dle nastavení).
  \item Každé slovo má nápovědu, která je dostupná na začátku hry.
  \item Hra podporuje vibrace při chybě (pokud je zapnuto).
\end{itemize}

\section{Práce s daty a uložení}

\subsection{Uložení nastavení a skóre}
Aplikace ukládá uživatelské nastavení (hudba, zvuk, vibrace, téma, limit) do \texttt{AsyncStorage}. Stejným způsobem se ukládá i skóre jednotlivých slov (maximální dosažený počet hvězd).

\subsection{Datové soubory}
\begin{itemize}
  \item \texttt{words.json} – obsahuje seznam slov rozdělených podle témat a obtížnosti.
  \item \texttt{buttons.json} – obsahuje metainformace o obtížnostech a tématech (barva, ikona).
  \item \texttt{imageMap.tsx} – mapuje konkrétní slova na výherní obrázky.
\end{itemize}

\section{Popis jednotlivých obrazovek}

\begin{itemize}
  \item \textbf{Hlavní menu (\texttt{index.tsx})} – vstupní obrazovka s tlačítky: Začít hru, Nastavení, Ukončit.
  \item \textbf{Nastavení (\texttt{settings.tsx})} – volby pro téma, zvuky, vibrace, limit, reset hry.
  \item \textbf{Herní menu (\texttt{menu.tsx})} – výběr obtížnosti a tématu.
  \item \textbf{Úrovně (\texttt{levels.tsx})} – seznam dostupných úrovní dle výběru.
  \item \textbf{Hra (\texttt{game.tsx})} – hlavní logika hry, ovládání písmeny, zobrazení šibenice, nápověda.
  \item \textbf{Postup (\texttt{progress.tsx})} – stromové zobrazení výsledků s koláčovým grafem.
  \item \textbf{Výhra (\texttt{win.tsx})} – animace, hvězdy, obrázek a možnost pokračování.
\end{itemize}

\section{Platformy a SDK}

\begin{itemize}
  \item \textbf{Platformy:} Android, iOS
  \item \textbf{Minimální SDK:} Expo SDK 50 (React Native 0.73)
  \item Aplikace funguje na zařízeních s Android 7.0+ a iOS 13+
  \item Optimalizováno pro telefony i tablety
\end{itemize}

\end{document}

