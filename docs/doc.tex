\documentclass[a4paper,12pt]{article}

\usepackage[czech]{babel}
\usepackage[a4paper,margin=2.5cm]{geometry}
\usepackage{hyperref}

\title{Dokumentace k projektu \textbf{Šibenice}}
\author{}
\date{\today}

\begin{document}

\maketitle

\section{Úvod}

Tento dokument slouží jako technická dokumentace k projektu \textbf{Šibenice}, což je mobilní hra určená pro platformy \textbf{Android} a \textbf{iOS}. 
Hra je postavena na technologii \textbf{React Native} s využitím frameworku \textbf{Expo} a poskytuje uživatelsky přívětivé rozhraní pro procvičování slovní zásoby z různých kategorií.

Hlavním cílem aplikace je vytvořit interaktivní a edukativní herní prostředí, které uživateli umožní hravou formou testovat a rozvíjet slovní zásobu. 
Hra zahrnuje různé obtížnosti, skórovací systém a možnosti přizpůsobení. 

Tato dokumentace pokrývá následující oblasti:
\begin{itemize}
    \item Architektura a technologie použité v aplikaci,
    \item Popis implementace a struktury kódu,
    \item Instrukce ke spuštění aplikace,
    \item Přehled funkcionalit a jejich využití,
    \item Ukázky obrazovek a jejich popis.
\end{itemize}

Cílem dokumentace je poskytnout podrobný přehled o vývoji aplikace a umožnit její rozšiřování nebo úpravy v budoucnu.

\section{Architektura aplikace}

Aplikace \textbf{Šibenice} je postavena na technologii \textbf{React Native} a využívá framework \textbf{Expo}, který zjednodušuje vývoj, testování a nasazení mobilních aplikací. Struktura projektu je navržena modulárně, aby byla umožněna snadná údržba a rozšiřitelnost funkcionalit.

\subsection{Modulární přístup}
Projekt je rozdělen do jednotlivých komponent, které pokrývají různé funkční části aplikace. Mezi klíčové moduly patří:
\begin{itemize}
    \item \textbf{Hlavní obrazovka (Menu)} – výběr tématu a obtížnosti,
    \item \textbf{Obrazovka s úrovněmi (Levels)} – seznam dostupných úrovní pro zvolené téma a obtížnost,
    \item \textbf{Herní obrazovka (Game)} – samotný průběh hry, zobrazení šibenice a hádání písmen,
    \item \textbf{Výherní obrazovka (Win)} – animace a skórování po úspěšném dokončení úrovně,
    \item \textbf{Nastavení (Settings)} – možnosti přizpůsobení hry, včetně zvuků, tématu a obtížnosti.
\end{itemize}

\subsection{Navigace v aplikaci}
Aplikace využívá systém navigace \textbf{Expo Router}, který umožňuje přehlednou správu jednotlivých obrazovek a přechod mezi nimi. Navigace je založena na hierarchickém uspořádání stránek:

\begin{itemize}
    \item \texttt{index.tsx} – hlavní stránka aplikace,
    \item \texttt{menu.tsx} – přechod na výběr tématu a obtížnosti,
    \item \texttt{levels.tsx} – seznam úrovní pro konkrétní téma a obtížnost,
    \item \texttt{game.tsx} – samotná hra,
    \item \texttt{win.tsx} – obrazovka s oznámením o výhře,
    \item \texttt{settings.tsx} – nastavení aplikace.
\end{itemize}

Každá z těchto obrazovek je samostatná komponenta, která komunikuje s ostatními částmi aplikace pomocí parametrů v navigaci.

\subsection{Ukládání dat}
Aplikace využívá \textbf{AsyncStorage} pro uchování uživatelských preferencí a skóre. Hlavní uložené informace zahrnují:
\begin{itemize}
    \item \textbf{Nastavení uživatele} – barevné téma, zvuky, vibrace, časový limit.
    \item \textbf{Skóre} – uložené nejlepší výsledky pro jednotlivá slova a úrovně.
\end{itemize}

\subsection{Použité knihovny}
Aplikace se opírá o několik knihoven pro usnadnění vývoje a vylepšení uživatelského rozhraní:
\begin{itemize}
    \item \texttt{react-native-vector-icons} – ikony pro vizuální prvky,
    \item \texttt{react-native-progress} – ukazatel průběhu ve hře,
    \item \texttt{react-native-confetti-cannon} – animace oslav při výhře,
    \item \texttt{expo-router} – správa navigace mezi obrazovkami.
\end{itemize}

\section{Hlavní funkcionality aplikace}

Aplikace \textbf{Šibenice} obsahuje několik klíčových funkcionalit, které umožňují interaktivní a dynamickou hratelnost. Tyto funkcionality zahrnují výběr témat a obtížností, správu skóre, časový limit a vizuální zpětnou vazbu.

\subsection{Výběr tématu a obtížnosti}

Při spuštění hry si uživatel nejprve zvolí herní téma a úroveň obtížnosti. Témata jsou rozdělena do několika kategorií, jako například:
\begin{itemize}
    \item \textbf{Zvířata},
    \item \textbf{Cestování},
    \item \textbf{Jídlo},
    \item \textbf{Auta},
    \item \textbf{Technologie},
    \item \textbf{Filmy},
    \item \textbf{Historie},
    \item \textbf{Věda} a další.
\end{itemize}

Obtížnosti jsou rozděleny na tři úrovně:
\begin{itemize}
    \item \textbf{Začátečník} – jednodušší slova s více nápovědami,
    \item \textbf{Pokročilý} – střední úroveň obtížnosti s méně nápovědami,
    \item \textbf{Expert} – nejtěžší úroveň, často se složitějšími slovy.
\end{itemize}

Výběr obtížnosti ovlivňuje herní mechaniku, například počet pokusů na uhodnutí slova.

\subsection{Průběh hry a pravidla}

Hráč hádá slovo zadáváním jednotlivých písmen. Pokud se písmeno nachází ve slově, je odhaleno na správných pozicích. Pokud ne, je hráči odečten jeden pokus. Hra je ukončena v těchto případech:
\begin{itemize}
    \item Hráč správně uhodne celé slovo – zobrazí se animace výhry a přidělené skóre.
    \item Hráč vyčerpá maximální počet nesprávných pokusů – zobrazí se obrazovka s neúspěchem.
    \item Vyprší časový limit (pokud je aktivován) – hráč automaticky prohrává kolo.
\end{itemize}

\subsection{Skórovací systém}

Každé slovo lze ohodnotit na základě počtu chyb. Skórování probíhá podle těchto pravidel:
\begin{itemize}
    \item \textbf{3 hvězdy} – žádná chyba,
    \item \textbf{2 hvězdy} – maximálně 2 chyby,
    \item \textbf{1 hvězda} – maximálně 4 chyby,
    \item \textbf{0 hvězd} – více než 4 chyby.
\end{itemize}

Skóre se ukládá pomocí \textbf{AsyncStorage} a při opětovném hraní stejné úrovně se započítá nejvyšší dosažené skóre.

\subsection{Nápovědy}

Pro usnadnění hry je k dispozici nápověda, která poskytuje krátký popis hádaného slova. Nápovědu lze zobrazit kliknutím na odpovídající tlačítko.

\subsection{Časový limit a pokusy}

Hra podporuje možnost nastavení časového limitu pro dokončení úrovně. Pokud je aktivován, odpočítává se čas a při jeho vypršení hra automaticky končí prohrou. Dále lze aktivovat omezený počet pokusů na hádání slova.

\subsection{Animace a vizuální efekty}

Aplikace využívá různé vizuální efekty pro zvýšení interaktivity:
\begin{itemize}
    \item Animace konfet při výhře (\textbf{ConfettiCannon}),
    \item Ukazatel průběhu (\textbf{react-native-progress}),
    \item Vibrace zařízení při nesprávné odpovědi (pokud je aktivována).
\end{itemize}

\section{Uživatelské rozhraní a navigace}

Aplikace \textbf{Šibenice} využívá moderní a přehledné uživatelské rozhraní založené na komponentách \textbf{React Native}. Rozhraní je optimalizováno pro mobilní zařízení s různými velikostmi obrazovek a podporuje přepínání mezi světlým a tmavým režimem.

\subsection{Hlavní obrazovky aplikace}

Aplikace se skládá z několika hlavních obrazovek, mezi kterými se uživatel může pohybovat prostřednictvím navigačního systému \textbf{Expo Router}:

\begin{itemize}
    \item \textbf{Úvodní obrazovka (index.tsx)} – zobrazuje logo aplikace a základní možnosti (spuštění hry, nastavení).
    \item \textbf{Hlavní menu (menu.tsx)} – umožňuje výběr tématu a obtížnosti.
    \item \textbf{Seznam úrovní (levels.tsx)} – zobrazuje dostupné úrovně s vizuální indikací dosaženého skóre.
    \item \textbf{Herní obrazovka (game.tsx)} – obsahuje interaktivní herní plochu s možností hádání slov.
    \item \textbf{Obrazovka výhry (win.tsx)} – zobrazí se po úspěšném dokončení úrovně.
    \item \textbf{Obrazovka nastavení (settings.tsx)} – umožňuje změnit téma, zvuk, vibrace a další nastavení.
\end{itemize}

\subsection{Hlavní menu}

Po spuštění aplikace se uživateli zobrazí hlavní menu, kde si může vybrat téma hry a úroveň obtížnosti. Menu je přehledně rozvržené do kategorií, které jsou barevně odlišeny:

\begin{itemize}
    \item Každé téma je reprezentováno ikonou a barvou (např. \textbf{Zvířata} – oranžová, \textbf{Technologie} – modrá).
    \item Obtížnosti mají vlastní barevné kódování: \textbf{Začátečník} (modrá), \textbf{Pokročilý} (oranžová), \textbf{Expert} (červená).
\end{itemize}

\subsection{Herní obrazovka}

Herní obrazovka je navržena tak, aby byla co nejintuitivnější:
\begin{itemize}
    \item Zobrazuje aktuální slovo jako sérii podtržítek, kde se postupně doplňují správně uhodnutá písmena.
    \item Pod hádaným slovem je abecední klávesnice, která umožňuje výběr písmen.
    \item Indikátor pokusů (v případě omezených pokusů) nebo odpočet času (pokud je aktivován časový limit).
    \item Možnost zobrazení nápovědy.
\end{itemize}

Při správném uhodnutí slova se zobrazí animace výhry a přesměrování na obrazovku s výsledkem.

\subsection{Obrazovka výsledku}

Po dokončení úrovně se uživateli zobrazí stránka s výsledkem:
\begin{itemize}
    \item Počet hvězd (hodnocení dle počtu chyb),
    \item Možnost pokračovat na další úroveň nebo se vrátit do menu,
    \item Krátká animace konfet pro zvýraznění výhry.
\end{itemize}

\subsection{Navigace mezi obrazovkami}

Navigace mezi obrazovkami probíhá prostřednictvím \textbf{Expo Router}, který umožňuje přehledné a efektivní řízení přechodů. Každá stránka aplikace je reprezentována samostatnou komponentou a mezi stránkami se přechází pomocí parametrů v URL.

Uživatel může navigovat:
\begin{itemize}
    \item \textbf{Tlačítky zpět} – návrat na předchozí stránku (např. zpět do menu),
    \item \textbf{Automatickými přechody} – po dokončení úrovně je hráč přesměrován na výsledkovou obrazovku,
    \item \textbf{Přímým výběrem} – uživatel si může kdykoliv vybrat jinou úroveň v seznamu úrovní.
\end{itemize}

\subsection{Přizpůsobení vzhledu}

Aplikace podporuje přepínání mezi světlým a tmavým režimem, které se nastavuje v sekci \textbf{Nastavení}. Volba se ukládá pomocí \textbf{AsyncStorage} a při dalším spuštění se automaticky obnoví.

\section{Technická implementace a struktura kódu}

Aplikace \textbf{Šibenice} je postavena na moderním technologickém stacku s využitím \textbf{React Native} a \textbf{Expo}. Kód je rozdělen do několika modulů, což umožňuje snadnou údržbu a rozšiřitelnost projektu.

\subsection{Adresářová struktura}

Projekt je organizován podle doporučených postupů pro aplikace v \textbf{Expo Router}. Hlavní složky jsou následující:

\begin{verbatim}
/
|-- assets/              % Obsahuje fonty, obrázky a zvuky
|-- components/          % Opakovaně použitelné React komponenty
|-- screens/             % Hlavní obrazovky aplikace
|   |-- Game.tsx        % Herní obrazovka
|   |-- Menu.tsx        % Výběr tématu a obtížnosti
|   |-- Levels.tsx      % Seznam úrovní
|   |-- Win.tsx         % Obrazovka výhry
|   |-- Settings.tsx    % Nastavení hry
|   |-- Index.tsx       % Úvodní obrazovka aplikace
|-- data/                % JSON soubory se slovy a nastavením
|   |-- words.json      % Slova podle témat a obtížností
|   |-- buttons.json    % Konfigurace tlačítek pro menu
|-- utils/               % Pomocné funkce a utility
|-- App.tsx              % Hlavní soubor aplikace
|-- package.json         % Konfigurace balíčků
\end{verbatim}

Tato struktura umožňuje jasné oddělení jednotlivých částí aplikace.

\subsection{Klíčové komponenty}

Každá část aplikace je implementována jako samostatná komponenta v souboru \texttt{.tsx}. Mezi nejdůležitější komponenty patří:

\subsubsection{Game.tsx – herní logika}

\begin{itemize}
    \item Obsahuje hlavní herní mechanismus.
    \item Načítá slova z JSON souboru podle zvoleného tématu a obtížnosti.
    \item Sleduje správné a nesprávné odpovědi.
    \item Zobrazuje šibenici a progres hráče.
    \item Při výhře nebo prohře přesměrovává hráče na odpovídající obrazovku.
\end{itemize}

\subsubsection{Menu.tsx – výběr tématu a obtížnosti}

\begin{itemize}
    \item Obsahuje seznam témat a obtížností, které jsou načítány z JSON souboru.
    \item Při výběru se parametry předávají na stránku \texttt{Levels.tsx}.
    \item Každé téma a obtížnost mají unikátní barvu a ikonu.
\end{itemize}

\subsubsection{Levels.tsx – seznam úrovní}

\begin{itemize}
    \item Načítá seznam úrovní na základě zvoleného tématu a obtížnosti.
    \item Indikuje dosažené skóre pomocí hvězdiček.
    \item Přesměrovává hráče na \texttt{Game.tsx} s odpovídající úrovní.
\end{itemize}

\subsubsection{Win.tsx – obrazovka vítězství}

\begin{itemize}
    \item Zobrazuje hráči dosažené skóre a efekt konfet.
    \item Poskytuje možnosti pokračování na další úroveň nebo návrat do menu.
\end{itemize}

\subsubsection{Settings.tsx – nastavení hry}

\begin{itemize}
    \item Umožňuje změnu tématu aplikace (tmavý/světlý režim).
    \item Nabízí možnost aktivace/deaktivace zvuků a vibrací.
    \item Umožňuje nastavení časového limitu pro hru.
    \item Všechna nastavení se ukládají pomocí \textbf{AsyncStorage}.
\end{itemize}

\subsection{Práce s daty}

Aplikace využívá soubory JSON pro uchování herních dat:

\begin{itemize}
    \item \textbf{words.json} – obsahuje slova, jejich kategorie a nápovědy.
    \item \textbf{buttons.json} – obsahuje seznam obtížností a témat s barvami a ikonami.
    \item \textbf{AsyncStorage} – ukládá skóre hráče a uživatelská nastavení.
\end{itemize}

\subsection{Příklad práce s AsyncStorage}

Kód pro načtení a uložení skóre hráče:

\begin{verbatim}
const loadScores = async () => {
    try {
        const savedScores = await AsyncStorage.getItem("score");
        if (savedScores) {
            setScoreData(JSON.parse(savedScores));
        }
    } catch (error) {
        console.error("Chyba při načítání skóre:", error);
    }
};

const saveScore = async (word: string, newScore: number) => {
    try {
        const oldScore = scoreData[word] ?? 0;
        const betterScore = Math.max(oldScore, newScore);
        const updatedScoreData = { ...scoreData, [word]: betterScore };

        await AsyncStorage.setItem("score", JSON.stringify(updatedScoreData));
        setScoreData(updatedScoreData);
    } catch (error) {
        console.error("Chyba při ukládání skóre:", error);
    }
};
\end{verbatim}

Tento mechanismus umožňuje uložení nejlepšího skóre pro jednotlivá slova.

\section{Testování, ladění a nasazení aplikace}

V této kapitole jsou popsány způsoby testování a ladění aplikace během vývoje a postupy potřebné pro její nasazení na mobilní platformy Android a iOS.

\subsection{Testování aplikace}

Testování aplikace probíhá v několika fázích:
\begin{itemize}
    \item \textbf{Vývojové testování} – průběžné testování funkcionalit během vývoje.
    \item \textbf{Automatizované testy} – kontrola správnosti funkcí pomocí testovacích knihoven.
    \item \textbf{Testování na fyzických zařízeních} – ověření chování aplikace v reálných podmínkách.
\end{itemize}

\subsubsection{Vývojové testování v Expo}
Během vývoje je aplikace spouštěna v Expo Developer Tools pomocí příkazu:

\begin{verbatim}
npx expo start
\end{verbatim}

Tento nástroj umožňuje:
\begin{itemize}
    \item Spouštění aplikace v emulátorech i na fyzických zařízeních (Android/iOS).
    \item Automatický reload při změnách kódu (\textit{Fast Refresh}).
    \item Přístup ke konzoli pro ladění chyb v kódu.
\end{itemize}

\subsubsection{Automatizované testy}

Aplikace může využívat testovací frameworky jako:
\begin{itemize}
    \item \textbf{Jest} – testování jednotlivých funkcí aplikace.
    \item \textbf{React Native Testing Library} – simulace interakce s uživatelským rozhraním.
\end{itemize}

Příklad základního testu pro ověření existence komponenty:

\begin{verbatim}
import { render } from "@testing-library/react-native";
import Game from "../screens/Game";

test("Zkontroluje renderování herní obrazovky", () => {
    const { getByText } = render(<Game />);
    expect(getByText("Hádej slovo")).toBeTruthy();
});
\end{verbatim}

\subsubsection{Testování na fyzických zařízeních}

Testování na reálných mobilních zařízeních se provádí pomocí:
\begin{itemize}
    \item \textbf{Expo Go} – umožňuje spustit aplikaci na mobilu bez nutnosti jejího sestavení.
    \item \textbf{Android Emulator / iOS Simulator} – simulace aplikace v prostředí dané platformy.
    \item \textbf{TestFlight} – distribuce testovacích verzí aplikace pro iOS.
\end{itemize}

\subsection{Ladění aplikace}

Expo poskytuje vestavěné nástroje pro ladění:
\begin{itemize}
    \item \textbf{React Developer Tools} – umožňuje inspekci komponent v reálném čase.
    \item \textbf{Debugging Console} – pro zobrazení chyb a výstupů z konzole.
    \item \textbf{Remote Debugging} – možnost ladění kódu přímo v prohlížeči.
\end{itemize}

\subsubsection{Ladění chyb pomocí výpisů do konzole}

V případě chyb lze použít \texttt{console.log()} pro diagnostiku:

\begin{verbatim}
useEffect(() => {
    console.log("Aktuální téma:", topic);
}, [topic]);
\end{verbatim}

Další možnosti ladění zahrnují použití breakpoints v Chrome DevTools.

\subsection{Nasazení aplikace}

Aplikace může být nasazena dvěma způsoby:
\begin{itemize}
    \item \textbf{Expo Managed Workflow} – snadné sestavení a distribuce pomocí Expo Cloud Build.
    \item \textbf{Eject do Bare Workflow} – přechod na nativní vývoj s možností integrace nativních modulů.
\end{itemize}

\subsubsection{Příprava na produkční sestavení}

Pro nasazení je nutné vytvořit optimalizovanou verzi aplikace:

\begin{verbatim}
eas build --platform android
eas build --platform ios
\end{verbatim}

Tento proces vygeneruje instalační soubory \texttt{.apk} nebo \texttt{.aab} pro Android a \texttt{.ipa} pro iOS.

\subsubsection{Publikace na Google Play a App Store}

Publikace aplikace probíhá v několika krocích:
\begin{enumerate}
    \item Registrace vývojářského účtu na \textbf{Google Play Console} nebo \textbf{Apple Developer Program}.
    \item Nahrání sestavené aplikace do příslušného obchodu.
    \item Vyplnění metadat (popis, obrázky, kategorie).
    \item Odeslání k recenzi a schválení.
\end{enumerate}

Pro usnadnění nasazení lze využít příkaz:

\begin{verbatim}
eas submit --platform android
\end{verbatim}

\section{Budoucí vylepšení a možné rozšíření}

Aplikace \textbf{Šibenice} poskytuje stabilní a dobře strukturovaný základ, který umožňuje její další rozšiřování. Tato kapitola se zabývá návrhy možných vylepšení, která mohou zvýšit uživatelský komfort a rozšířit funkcionalitu aplikace.

\subsection{Možná rozšíření funkcionality}

\subsubsection{Online multiplayer mód}

Jednou z nejzajímavějších možností rozšíření je implementace \textbf{multiplayer režimu}, kde by hráči mohli hrát proti sobě v reálném čase. Tento režim by mohl zahrnovat:
\begin{itemize}
    \item Připojení k online serveru a možnost pozvat přátele ke hře.
    \item Systém bodování a žebříčku nejlepších hráčů.
    \item Možnost hraní proti náhodným soupeřům.
\end{itemize}

Technologicky by multiplayer mohl být postaven na \textbf{WebSockets} nebo \textbf{Firebase Realtime Database} pro synchronizaci hry mezi hráči.

\subsubsection{Personalizovaný tréninkový režim}

Aplikace by mohla obsahovat \textbf{tréninkový režim}, kde by hráči mohli procvičovat konkrétní slova nebo tematické okruhy dle vlastního výběru. Možné funkce:
\begin{itemize}
    \item Výběr obtížnosti na základě historie hráče.
    \item Statistické vyhodnocení pokroků ve slovní zásobě.
    \item Možnost opakování slov, která hráč neuhodl.
\end{itemize}

\subsubsection{Vylepšený systém nápovědy}

Současná nápověda poskytuje pouze textový popis slova. Možná vylepšení:
\begin{itemize}
    \item Zobrazení obrázkové nápovědy (přidání obrázků k jednotlivým slovům).
    \item Možnost odhalení prvního nebo posledního písmene jako součást nápovědy.
    \item Systém odemykatelných nápověd na základě nasbíraného skóre.
\end{itemize}

\subsubsection{Tématické události a výzvy}

Aplikace by mohla nabízet \textbf{časově omezené výzvy}, které by motivovaly hráče k pravidelnému hraní. Možné implementace:
\begin{itemize}
    \item Speciální týdenní/měsíční výzvy se zaměřením na určitá témata.
    \item Sezónní události (např. vánoční, letní edice s novými slovy).
    \item Odměny za dokončení výzev, například nové vizuální motivy aplikace.
\end{itemize}

\subsection{Vylepšení uživatelského rozhraní a designu}

\subsubsection{Možnosti personalizace vzhledu}
Aplikace by mohla umožnit větší \textbf{personalizaci vzhledu} podle preferencí uživatele:
\begin{itemize}
    \item Možnost výběru barevného schématu aplikace.
    \item Různé vizuální styly šibenice.
    \item Možnost změny fontu a velikosti písma.
\end{itemize}

\subsubsection{Animace a efekty}
Zlepšení vizuálních efektů by mohlo aplikaci zpříjemnit pro hráče:
\begin{itemize}
    \item Plynulé animace při zadávání písmen.
    \item Efekty zvýraznění při správném/špatném odhadu.
    \item Animovaný progres indikující postup ve hře.
\end{itemize}

\subsection{Lepší integrace s externími službami}

\subsubsection{Ukládání progresu na cloud}
Možná implementace ukládání dat do cloudu, aby hráč nepřišel o svůj postup:
\begin{itemize}
    \item Integrace s \textbf{Firebase} nebo jinou cloudovou službou pro synchronizaci dat mezi zařízeními.
    \item Přihlášení pomocí Google/Facebook pro zálohování progresu.
\end{itemize}

\subsubsection{Podpora více jazyků}
Aplikace by mohla být rozšířena o podporu více jazyků:
\begin{itemize}
    \item Možnost přepnutí mezi různými jazyky hry.
    \item Slovníky přizpůsobené pro různé jazykové mutace.
\end{itemize}

\subsection{Optimalizace výkonu}

Pro lepší výkon aplikace by bylo možné:
\begin{itemize}
    \item Zredukovat velikost balíčku odstraněním nepotřebných knihoven.
    \item Optimalizovat načítání slov a assetů pro rychlejší odezvu aplikace.
    \item Použití \textbf{Hermes Engine} pro efektivnější běh aplikace na Androidu.
\end{itemize}

\section{Závěr}

Tato dokumentace poskytuje ucelený přehled o vývoji, architektuře, funkcionalitě a možnostech rozšíření mobilní hry \textbf{Šibenice}. Aplikace byla navržena jako interaktivní a edukativní hra, která umožňuje hráčům procvičovat slovní zásobu z různých tematických oblastí.

Hlavní přínosy aplikace:
\begin{itemize}
    \item Podpora multiplatformního vývoje pro \textbf{Android} a \textbf{iOS}.
    \item Snadná rozšiřitelnost díky použití \textbf{React Native} a \textbf{Expo}.
    \item Intuitivní uživatelské rozhraní a přizpůsobitelná herní mechanika.
    \item Ukládání herního progresu pomocí \textbf{AsyncStorage}.
    \item Možnost budoucích vylepšení, včetně online multiplayeru, cloudového ukládání a širší personalizace.
\end{itemize}

\subsection{Shrnutí vývoje}

Během vývoje byly použity moderní technologie, které usnadňují správu kódu a optimalizují výkon aplikace. Klíčové části implementace zahrnovaly:
\begin{itemize}
    \item Implementaci navigace mezi obrazovkami pomocí \textbf{Expo Router}.
    \item Použití knihovny \textbf{react-native-vector-icons} pro vizuální zlepšení.
    \item Integraci zvukových efektů a animací pro lepší uživatelský zážitek.
    \item Strukturované zpracování dat uložených ve formátu \textbf{JSON}.
    \item Možnost přepínání mezi světlým a tmavým režimem.
\end{itemize}

\subsection{Další kroky}

V současné verzi aplikace je základní herní funkcionalita plně funkční. Další vývoj by se mohl zaměřit na:
\begin{itemize}
    \item Implementaci online režimu pro soutěžení hráčů v reálném čase.
    \item Rozšíření databáze slov a přidání nových tematických okruhů.
    \item Vylepšení vizuálních efektů a animací.
    \item Integraci cloudového ukládání progresu pomocí \textbf{Firebase}.
\end{itemize}

\subsection{Závěrečné hodnocení}

Aplikace \textbf{Šibenice} představuje kvalitní základ pro mobilní vzdělávací hru, která kombinuje zábavu s učením. Díky flexibilní architektuře je možné ji dále rozšiřovat a přizpůsobovat novým požadavkům uživatelů. Tento dokument by měl sloužit jako průvodce pro budoucí údržbu a rozvoj projektu.

\end{document}